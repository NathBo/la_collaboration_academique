\documentclass{article}
\usepackage[T1]{fontenc} % Add this line
\usepackage{lmodern}
\usepackage[french]{babel}
\usepackage{amsmath}
\usepackage{amssymb}
\usepackage{graphicx}
\usepackage{geometry}

\geometry{
    left=2cm,
    right=2cm,
    top=4cm,
    bottom=4cm
}

\title{Fonctions OCaml utiles en TD d'OS}
\author{Nathan Boyer}

\begin{document}

\maketitle

\begin{table}[htbp]
    \centering
    \begin{tabular}{|p{4cm}|p{3cm}|p{7cm}|} % Adjusted column widths
        \hline
        Fonction & Signature & Effet \\ 
        \hline 
        handle\_unix\_error f x & (a->b) -> a -> b & Applique f à x et renvoie le résultat, si le résultat est une erreur, ça écrit l'erreur et ça interrompt avec le code de sortie 2 \\ 
        \hline 
        stats & type & type enregistrement qui contient plein d'infos sur un fichier, notamment st\_ino le numero de noeud et st\_dev le numere de partition \\ 
        \hline 
        stat f & string -> stats & renvoie les stats du fichier f \\ 
        \hline 
        dir\_handle & type & descripteur d'un dossier ouvert \\ 
        \hline 
        opendir d & string -> dir\_handle & Renvoie un dir\_handle du dossier d \\ 
        \hline 
        readdir d & dir\_handle -> string & Renvoie la prochaine entree du dossier d, leve End\_Of\_File si on en est a la fin \\ 
        \hline 
        closedir d & dir\_handle -> unit & Ferme le dossier \\ 
        \hline 
        file_perm & type (=int) & les permissions d'un fichier \\ 
        \hline 
    \end{tabular}
    \caption{Tableau des fonctions OCaml}
    \label{tab:tableau1} 
\end{table}

\end{document}
