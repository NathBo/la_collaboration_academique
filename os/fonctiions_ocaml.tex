\documentclass{article}
\usepackage[T1]{fontenc}
\usepackage{lmodern}
\usepackage[french]{babel}
\usepackage{amsmath}
\usepackage{amssymb}
\usepackage{graphicx}
\usepackage{geometry}
\usepackage{longtable} % Include the longtable package

\geometry{
    left=2cm,
    right=2cm,
    top=4cm,
    bottom=4cm
}

\title{Fonctions OCaml utiles en TD d'OS}
\author{Nathan Boyer}

\begin{document}

\maketitle


\begin{longtable}{|p{4cm}|p{3cm}|p{7cm}|} 
\hline
Fonction & Signature & Effet \\ 
\hline 
\endfirsthead

\multicolumn{3}{c}%
{{\bfseries \tablename\ \thetable{} -- Suite de la page précédente}} \\
\hline
Fonction & Signature & Effet \\ 
\hline 
\endhead

\hline \multicolumn{3}{|r|}{{Suite à la page suivante}} \\ \hline
\endfoot

\hline
\endlastfoot

handle\_unix\_error f x & (a->b) -> a -> b & Applique f à x et renvoie le résultat, si le résultat est une erreur, ça écrit l'erreur et ça interrompt avec le code de sortie 2 \\ 
\hline 
stats & type & type enregistrement qui contient plein d'infos sur un fichier, notamment st\_ino le numero de noeud et st\_dev le numere de partition \\ 
\hline 
stat f & string -> stats & renvoie les stats du fichier f \\ 
\hline 
dir\_handle & type & descripteur d'un dossier ouvert \\ 
\hline 
opendir d & string -> dir\_handle & Renvoie un dir\_handle du dossier d \\ 
\hline 
readdir d & dir\_handle -> string & Renvoie la prochaine entree du dossier d, leve End\_Of\_File si on en est a la fin \\ 
\hline 
closedir d & dir\_handle -> unit & Ferme le dossier \\ 
\hline 
file\_perm & type (=int) & les permissions d'un fichier \\ 
\hline 
openfile & string -> open\_flag list -> file\_perm -> file\_descr & ouvre le fichier, 3eme argument c les perms qu'on donne au fichier au cas où on le crée \\ 
\hline 
lseek & file\_descr -> int -> seek\_command -> int & définit la position actuelle du descripteur de fichier et renvoie l'offset résultant (mais on s'en fout du résultat, utiliser ignore)
la seek\_command c SEEK\_SET pour depuis le debut (yen a 2 autres aussi) \\ 
\hline 
read fd buffer start length & file\_descr -> bytes -> int -> int -> int & lit len octets du descripteur fd, en les stockant dans la séquence d'octets buf, à partir de la position pos dans buf. Retourne le nombre d'octets effectivement lus \\ 
\hline
bytes & type & comme un string mais mutable \\ 
\hline 
String.create n & int -> bytes & renvoie une nouvelle séquence d'octets de longueur n. La séquence n'est pas initialisée et contient des octets arbitraires \\ 
\hline 
fork & unit -> int & Crée un nouveau processus. L'entier retourné est 0 pour le processus enfant, le pid du processus enfant pour le processus parent \\ 
\hline 
execv prog args & string -> string array -> 'a & exécute le programme dans le fichier prog, avec les arguments args, et l'environnement du processus en cours. Ces fonctions execv* ne renvoient jamais rien : en cas de succès, le programme actuel est remplacé par le nouveau \\ 
\hline 
execvp prog args & string -> string array -> 'a & Identique à Unix.execv, sauf que le programme est recherché dans le chemin d'accès. \\ 
\hline 
chdir & string -> unit & change le repertoire de travail \\ 
\hline 
getcwd & renvoie le repertoire de travail & Effet \\ 
\hline 
getenv & string -> string & renvoie valeur d'une variable d'environnement (genre HOME) \\ 
\hline 
signal\_behavior & Signal\_default | Signal\_ignore | Signal\_handle of f & Signal\_default = abort \\ 
\hline 
Sys.set\_signal & int -> signal\_behavior -> unit & Définit le comportement du système à la réception d'un signal donné. Le premier argument est le numéro du signal \\ 
\hline 
pipe & unit -> file\_descr * file\_descr & Créer un pipe. Le premier composant du résultat est ouvert à la lecture, c'est la sortie du pipe. Le deuxième composant est ouvert en écriture, c'est l'entrée du pipe \\ 
\hline 
dup2 scr dst & file\_descr -> file\_descr -> unit & duplique src vers dst, en fermant dst si déjà ouvert \\ 
\hline 
gethostbyname & string -> host\_entry & Trouver une entrée dans les hosts avec le nom donné \\ 
\hline 
sockaddr & ADDR\_UNIX of string | ADDR\_INET of inet\_addr * int & Le type des addresses de sockets \\ 
\hline 
socket & socket\_domain -> socket\_type -> int -> file\_descr & Crée une nouvelle socket dans le domaine donné, et avec le type donné (SOCK\_STREAM c bien). Le troisième argument est le type de protocole ; 0 sélectionne le protocole par défaut pour ce type de sockets \\ 
\hline 
connect & file\_descr -> sockaddr -> unit & Connecte une socket à une addresse \\ 
\hline 
send & file\_descr -> bytes -> int -> int -> msg\_flag list -> int & Envoyer data a travers connected socket \\ 
\hline 
recv & file\_descr -> bytes -> int -> int -> msg\_flag list -> int & Recevoir data a travers connected socket \\ 
\hline 
in/out\_channel\_of\_descr & file\_descr -> in/out\_channel & Création d'un canal d'entrée/sortie lisant le descripteur donné \\ 
\hline 
bind & file\_descr -> sockaddr -> unit & Bind une socket à une addresse \\ 
\hline 
listen & file\_descr -> int -> unit & Met en place un socket pour recevoir les demandes de connexion. L'argument entier est le nombre maximal de demandes en attente. \\ 
\hline 
accept & file\_descr -> file\_descr * sockaddr & Accepte les connexions sur le socket donné. Le descripteur renvoyé est un socket connecté au client ; l'adresse renvoyée est l'adresse du client connecté \\ 
\hline 
\caption{Tableau des fonctions OCaml}
\label{tab:tableau1} 
\end{longtable}

\end{document}
